% 独自のコマンド

% ■ アブストラクト
%	¥begin{jabstract} 〜 ¥end{jabstract}	:日本語のアブストラクト
%	¥begin{eabstract} 〜 ¥end{eabstract}	:英語のアブストラクト

% ■ 謝辞
%	¥begin{acknowledgment} 〜 ¥end{acknowledgment}

% ■ 文献リスト
%	¥begin{bib}[100] 〜 ¥end{bib}


¥newif¥ifjapanese

¥japanesetrue	% 論文全体を日本語で書く(英語で書くならコメントアウト)

¥ifjapanese
	¥documentclass[11pt]{jreport}
	¥renewcommand{¥bibname}{参考文献}
	¥newcommand{¥acknowledgmentname}{謝辞}
¥else
	¥documentclass[11pt]{report}
	¥newcommand{¥acknowledgmentname}{Acknowledgment}
¥fi
¥usepackage{ascmac}
¥usepackage{graphicx}
¥usepackage{multirow}
¥usepackage{ylab_thesis}


¥bindermode	% バインダ用余白設定

% 日本語情報(必要なら)
¥jclass	{卒業論文}							% 論文種別
¥jtitle		{Android端末を用いたバランスボードシステムの提案}			% タイトル。改行する場合は¥¥を入れる
¥juniv		{慶應義塾大学}						% 大学名
¥jfaculty	{環境情報学部 環境情報学科}				% 学部、学科
¥jauthor	{重田 瑞希}						% 著者
¥jadvisor	{増井 俊之}{教授}					% 指導教官、形式は『{名前}{肩書}』
¥jhyear	{25}								% 平成○年度
¥jsyear	{2013}							% 西暦○年度
%¥jkeyword	{¥LaTeX、テンプレート、卒業論文}			% 論文のキーワード


¥begin{document}

¥jmaketitle		% 表紙(日本語)、不要ならコメントアウト
%¥emaketitle		% 表紙(英語)、不要ならコメントアウト

¥include{00_abstract}	% アブストラクト。要独自コマンド、include先参照のこと

¥tableofcontents	% 目次
¥listoffigures		% 図目次

¥pagenumbering{arabic}

¥include{01}	% 本文1
¥include{02}	% 本文2
¥include{03}	% 本文3
¥include{04}	% 本文4
¥include{05}	% 本文4
¥include{06}	% 本文4

¥include{90_acknowledgment}	% 謝辞。要独自コマンド、include先参照のこと
¥include{91_bibliography}	% 参考文献。要独自コマンド、include先参照のこと
¥appendix
¥include{92_appendix}		% 付録

¥end{document}

